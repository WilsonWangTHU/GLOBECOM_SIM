\documentclass[conference]{IEEEtran}
\makeatletter
\newcommand{\rmnum}[1]{\romannumeral #1}
\newcommand{\Rmnum}[1]{\expandafter\@slowromancap\romannumeral #1@}
\makeatother

\usepackage{epsfig}
\usepackage{amsopn}
\usepackage{subfigure}
\usepackage{cite}
\ifCLASSINFOpdf

\else
\fi
\usepackage[cmex10]{amsmath}

\interdisplaylinepenalty = 2500

\usepackage{algorithmic}
\usepackage{algorithm}
\hyphenation{op-tical net-works semi-conduc-tor}


\begin{document}
\title{Access Points Selection in Super WiFi Network Powered by Solar Energy Harvesting}

\author{\authorblockN{Tingwu~Wang\authorrefmark{1},
	Chunxiao~Jiang\authorrefmark{1}, 
	Yan~Chen\authorrefmark{2}, 
	Yong~Ren\authorrefmark{1}, and 
	K. J. Ray~Liu\authorrefmark{2}}
	\small\authorblockA{\authorrefmark{1}
		Department of Electronic Engineering, Tsinghua University, Beijing, 100084, P. R. China\\
          \authorrefmark{2}Department of Electrical and Computer Engineering, 
		  University of Maryland, College Park, MD 20742, USA\\
        E-mail: wtw12@mails.tsinghua.edu.cn, \{jchx, reny\}@tsinghua.edu.cn, \{yan, kjrliu\}@umd.edu}}
\maketitle

\begin{abstract}
The announced yet not realized Super Wi-Fi Network is planning to enable Internet access in a nation-wide area.
The infrastructure deployment for Super Wi-Fi is a crucial problem. 
As traditional cable-connected power supply becomes impractical or costful, Energy Harvesting (EH) becomes a promising method.
In an EH powered network, previous models studying access point (AP) selection become invalid, 
as a result of the drastic changing number of users and battery states.
In this paper, we study the AP selection problem from the users' perspective 
and proposed a practical as well as efficient framework for deploying EH powered Super Wi-Fi network.
In our work, the Secondary User (SU) tries to connect to different BS with varing number of Primary User (PU)
and battery state, exploiting downlink transmission opportunities.
In order to formulate the problem, instead of assuming full knowledge of the system,
we used Partially Observable Markov Decision Process (POMDP) to incorporate the 
real world uncertainty. 
Simulation results show that our method remakably outperforms the random, myopic 
and the traditionly widely used listen-before-talk algorithm.
And in order to reduce the complexity of the POMDP, 
we proposed an algorithm that is sub-optimal yet able to accommendate to environmental changes.
\end{abstract}
\IEEEpeerreviewmaketitle
\section{Introduction}
% ------------------------------------------------------------------------------------------- %
% background
The battle to expand the coverage area of wireless access keeps pushing forward, 
and the Federal Communications Commission published Super Wi-Fi proposal, 
aiming to make use of lower-frequency white spaces between television channel frequencies 
and create a national wide wireless network. 
While there are a few successful experimental deployments of Super Wi-Fi, 
the task of building a nation wide network comes across many obstacles.
Traditional cable-based backhaul and energy supply system are, first of all, 
costfull considering the labor for deploying and maintaining the network in various situations, 
and secondly, sometimes impossible in extreme, complex and even dangerous environments.
Despite all the difficulties, there are solutions. 
The wireless backhual has been proven to be effective \cite{} as a replacement for cable backhaul.
And the fast developing EH technology provides an ideal solution for the power supply problem.
EH could support wireless network service with adequate ambient energy,
making use of piezoelectric, thermal, solar energy etc.\\
% ------------------------------------------------------------------------------------------- %
% previous work
Clearly, to make use of the ambient energy, new protocols and modification are needed for the current wireless standards,
as some preliminary studies pointed out.\cite{}\\
% ------------------------------------------------------------------------------------------- %
% our work
However, all the existing works on the AP selection or single AP access strategy 
have at least one of the following drawbacks.
% 	first
Firstly, most of them assume a infinite and always sufficient battery supply. 
As mentioned above, the infinite battery supply is not practical in building the Super Wi-Fi network.
In deploying wide range wireless network, AP with limited battery and constantly energy exhaust is unavoidable.
% 	second
Secondly, many of the work do not consider the fact that the number of user and the remaining battery is changing quickly.
Learning based algorithm may perform disasterly bad when they are learning parameters determined by system states that 
are changing rapidly, causing merely oscillating learning parameter results and low performance. 
Or more often the case, the algorithm suffers because of wrong belief of the states.
% 	third
Thirdly, the influence of the desicion maker, i.e. the SU, is neglected.
When SU makes a decision, it inevitebly affects the state transition. 
The desicion maker could cause energy exhaust if it performs short-sighted strategy,
causing the PUs as well as himself starve or too unambitious causing energy waste in finite battery storage.
The collision and lower SINR could also be triggered by the desicion maker.
%	fourth
Fourthly, the full knowledge of the system is unrealistic.
The assumption that the desicion maker has the full knowledge will make the model simple, 
but will make the implementation in real life irrealizable.
% conclusion about our model
In this work, we propose an algorithm that is able to overcome the aformentioned drawbacks.
And although our work focuses on solar energy harvesting,
it is clear that the conclusion and the algorithm could be generalize to any AP selection problem in EH powered network.
% ------------------------------------------------------------------------------------------- %
% organization
The rest of this paper is organized as follow. 
We describe the system model In Section \Rmnum{2}. 
And in Section \Rmnum{3}, the POMDP formulation is presented, 
showing how we abstract the continous battery state and user state to 
controlable POMDP states and draws the optimal desicion.
In Section \Rmnum{4}, we evaluate the performance of our proposed approach with several famous traditional algorithm. 
Finally, Section \Rmnum{5} draws the conclusion.
% ------------------------------------------------------------------------------------------- %
% ----------------------------end of introduction-------------------------------------------- %
% ------------------------------------------------------------------------------------------- %
\section{System Model}
In this section, before we illustrate the proposed POMDP algorithm, 
we describe the system model of a wireless time-slotted access process in details 
As shown in Fig1., like in other wireless network, 
the decision maker could observe mutiple BSs, the number of which is denoted as \(N_A\).
For each BS, there is a maximum number of users that it could serve simultaneously due to limited spectrum, 
as overcrouded user could not maintain its receiving SINR. 
We use the notation \(N_U\) to represent the maximum number of users.
Thus, the user state in each BS could be denoted as \(S_U = i,\, i = \,0,\,1,\,...,\,N_U\).
Different from the discrete user state, the battery could be any continous value between \(0\) 
and the maximum battery value \(B_M\). The \(B_M\) and \(N_U\) for different BS could be different.\\
\subsection{Access and Observation Model}
During each time slot, a certain quantity of energy, denoted as \(E_H\) is harvested, 
and a certain number of energy \(E_T\) is consumed during transmission of wireless signals.
Between adjacent time slots,
the number of PUs is changed after their service request is fullfilled or the battery exhausts or shortage.
And thus a revised birth and death process is used to describe the process of user behavior. 
The transition probability of number of PUs is calculated as,
\begin{align}\label{formula1} 
&\zeta\left(S_U'| S_U, Q_B, \Phi\right) = \nonumber\\
&\begin{cases}
	\lambda, &\mbox{if $Q_B$ is enough and $S_U' = S_U + 1$,}\\
	\mu S_U, &\mbox{if $Q_B$ is enough and $S_U' = S_U - 1$,}\\
	I_0\left(S_U'\right), &\mbox{if $Q_B$ is insufficient,}\\
	0, &\mbox{otherwise.}\\
\end{cases}
\end{align}
In the above equation, \(\lambda\) is the birth rate and the \(\mu' = \mu S_U\) is the death rate of the process.
Note that when the battery is exhausted, all the users are droped, 
which is represented by a indication function \(I_0\left(S_U'\right)\).
The user number in next time slot is zero. 
Whether the \(Q_B\) is adequate or not is determined by \(S_U\), \(Q_B\) as well as the SU's decision, \(\Phi\). 
And it is illustrated in the coming physical battery subsection.
At the start of each time slot, the SU could either decide to access or sense one of the BSs within its range. 
When the SU decide to access the \(i^{th}\) BS, the action is \(\Phi_{a}^i\).
It sends a request signal to the chosen BS.
And if the BS has the requested battery, it will activate SU's transmission in this time slot.
At the end of the transmission, the BS would attach the system state of this BS to the PU.
But there is also risk that the SU fails because of low power or waste energy because of collision.
In this case, the BS will reject the request and send back a short message informing the BS of the BS's system state.
So in order to save time and energy, a more conservative idea is to sense the BS, i.e. \(\Phi_{s}^i\).
When the BS receives a hello sense signal, 
it also transmits short its system state to the user with neglectable energy consumption.
\subsection{Physical Battery Model}
The harvested energy \(E_T\) is determined by the environmental parameters, 
which, in our case, is the sun light density, and the battery volume.
Gaussian model has been widely used in previous work as a result of its ability to characterize the sun light,
which has a slow average density swift and fast small fluctuate \cite{}.
In a small time slot, denoted as \(T_L\), the average light density remains unchanged. 
The light density \(W_e\) is Gaussian distributed, \(\mathcal{N}\left(x;\mu_S,\sigma\right)\).
The harvested energy is \(E_H = W_eJ_{op}V_{op}\Omega_ST_L\eta\), 
where the \(J_{op}\) and \(V_{op}\) is the optimal operating point of the existing harvesting devices \cite{}, 
the \(\Omega_S\) is the size of the sun light panel and the \(\eta\) is the harvesting efficiency.
The consumed energy is adjusted to make sure transmission with certain SINR.
It is important to note that from the BS perspective, current feedback-based iteration algorithm
that evolves several time slots 
like Inner Loop Power Management is not valid as the system state changes during the time slots.
Here a static power management is implemented. 
The power consumption in a BS is determined by number of users, which is \(E_T = \Upsilon_T(S_U, Q_B, \Phi)\).
In most cases, the access of SU could be regarded as an extra PU for the BS, i.e., for BS \(i\),
\(\Upsilon_T(S_U, Q_B, \Phi = \Phi_{a}^{i}) = \Upsilon_T(S_U + 1, Q_B, \Phi \ne \Phi_{a}^i)\).
When the required battery is more than the BS's remaining battery,
the users with the higher priority is served under the best effort principle.
In our case, the SU has the lowest priority.
Now we could define that event that \(Q_B\) is adequate as \(Q_B- \Upsilon_T(S_U, Q_B, \Phi) \leq 0\).
Thus the battery in the next time slot could be expressed as follow,
\begin{equation}
	Q_B' = \mbox{min}\{Q_B + E_H - E_T, B_M.\}
\end{equation}
\section{POMDP AP selection}
In Section \Rmnum{2} we describe the system model, it is clear that decision has to carefully chosed.
There are several factors to be consider: the tradeoff between immediate reward and the long term reward and 
the uncertainty of system state during decision making.
POMDP, which is recently developing as a strong tool to deal with desicion making under uncertainty,
come as a perfect solution to this system. 
However, before we could formulate our proposed algorithm, it is important to define our system state,
and, before that, discretize the battery state.
\subsection{Battery Transition}
In the implementation of EH powered BS, the battery is a continous value. 
And in a POMDP, the states have to discrete. 
Intutively, we could use more battery states to represent the same battery volume, 
but this bring much increase in the complexity of POMDP.
As the fluctuation of the battery quantity is quasi-static after many time slots,
in our work, we use the quasi static method \cite{} to convert 
the continous battery quantity into \(N_B\) discrete battery states,
denoted as \(S_B\). \(S_B = \lfloor Q_B N_B / B_M \rfloor\). \(S_B\) could take values from \(1,\,2,\,...,\,N_B - 1\).
If we denote the discrete battery state change between time slot as \(\Delta_B\), 
then the probability could be computed as follow, where we first define a event for convenience. 
\begin{equation}
	\xi_j := \Big\{\frac{j}{N_B}\leq\frac{E_H - E_T}{B_M} \leq \frac{j+1}{N_B}\Big\}
\end{equation}
\begin{align}&\mbox{Pr}\left(\Delta_B = i |\Phi, E_H, \xi_j \right)=\nonumber\\
&\begin{cases} \frac{\left(E_H - E_T\right)N_B - jB_M}{B_M}, &\mbox{$i = j + 1$}\\
\frac{\left(j+1\right)B_M -\left(E_H - E_T\right)N_B} {B_M}, &\mbox{$i = j$}\\
0, &\mbox{otherwise.}\\
\end{cases}
\end{align}
In the equation, as mentioned in previous section \(E_T = \Upsilon_T(S_U, Q_B, \Phi)\).
The situation where the \(E_H - E_T \le 0\) is simply doing the mirror computation of the above equation.
When we consider the Gaussian distributed light density, the actual battery transition is then,
\begin{equation}\label{battery}
\begin{aligned}
	&\mbox{Pr}\left(\Delta_B = i |\Phi\right) = \\
	&\int_{\frac{iB_M}{N_B} + E_T}^{\frac{\left(i+1\right)B_M}{N_B} + E_T}
	\mbox{Pr}\left(\Delta_B = i |\Phi, E_H, \xi_i\right) \mathcal{N}\left(E_H;\bar{\mu_S},\bar{\sigma}\right) dE_H+\\
	& \int_{\frac{\left(i-1\right)B_M}{N_B} + E_T}^{\frac{iB_M}{N_B} + E_T} 
	\mbox{Pr}\left(\Delta_B = i |\Phi, E_H, \xi_{i-1}\right) \mathcal{N}\left(E_H;\bar{\mu_S},\bar{\sigma}\right) dE_H.\\
\end{aligned}
\end{equation}
In the equation, the mean and variance of the Gaussian distribution are scaled accordingly after mutiplication.
After defining the battery change probability, we could compute the system transition probability.
\subsection{System Transition Probability}
When the user transition probability and battery transtion probability is computable,
we could focus on the formulation of POMDP.
Note that the POMDP state is the system state, which contains state information of all BS. 
To make the following math more readable and flexible,
two equivalent notation of system state is used simultaneously, i.e. 
the centralized system state \(S\) and the decentralized system state \(S_d = \{S_B^1,\,S_U^1,\,...,\,S_B^{N_A},\,S_U^{N_A}\}\).
\(S = 1,\,2,\,...\,N_S\), where \(N_S = \left(N_BN_U\right)^{N_A}\) is the number of system state.
For the transition probability for a single BS, 
the probability of \(S_U'\) and \(S_B'\) in the next slot are conditionally independent 
given the current \(S_U\), \(S_B\) and \(\Phi\).
\begin{equation}
\begin{aligned}
	\mbox{P}\left(S_U',S_B'|S_U,S_B,\Phi\right) =
	\zeta\left(S_U'|S_U, S_B, \Phi\right) \delta\left(S_B'|S_U, S_B, \Phi\right)\\
\end{aligned}
\end{equation}
The user transition \(\zeta\left(S_U'|S_U, S_B, \Phi\right)\) is given in equation \eqref{formula1}.
And the battery transtion is calculated based on the equation \eqref{battery}, which is shown as followed.
\begin{align}
	&\delta\left(S_B'|S_U, S_B, \Phi\right)\nonumber\\
	= &
	\begin{cases} 
		\mbox{Pr}\left(\Delta_B = S_B' - S_B|\Phi \right) &\mbox{if $S_B' \le N_B - 1$,}\\
		\sum_{\Delta_B = S_B' - S_B}^{\Delta_B = \Delta_B^{Max}}\mbox{Pr}\left(\Delta_B|\Phi\right), 
		&\mbox{if $S_B' = N_B - 1$.}\\
\end{cases}
\end{align}
In the case of \(S_B'=N_B - 1\), the battery is full, meaning that the battery overflows. 
As the volume of the battery is limited, all the extra battery is abondoned.
The notation \(\Delta_B^{Max}\) is used to simplify our calculation, 
as the \(\mbox{Pr}\left(\Delta_B|\Phi\right)\) decreases with the \(\Delta_B\), 
and usually when the \(\Delta_B\) reaches the \(5\) or \(6\),
the probability reaches zero by \(4\) decimals in the numerical results.
The overall transition is computed by multiply the transition probability of all the BSs.
\begin{align}\label{transition}
	T\left(S'|S,\Phi\right) = \prod_{i = 1}^{i = N_A}\mbox{P}\left(S_B^{i,'}, S_U^{i,'}|S_U^i, S_B^i, \Phi\right)
\end{align}
\subsection{POMDP Optimal Algorithm}
In POMDP formulation, the SU only has the partial knowledge of the system.
At the end of each slot, the SU could get the BS's system state, 
for the \(S_B^O\) and \(S_U^O\) which we use \(\mathcal{O}\) to denote.
The observation probability function given the actual BS state in the next time slot is denoted as follow,
\(S_B^{t,'}\) and \(S_U^{t,'}\) are the system state of the BS \(t\) the SU senses or tries to access.
\begin{align}
	&\psi\left(O|S_U^{t,'}, S_B^{t,'}\right) = 
	\begin{cases} 
		1 &\mbox{if $S_B^O = S_B^{t,'}, S_U^O=S_U^{t,'}$,}\\
		0 &\mbox{otherwise.}\\
\end{cases}
\end{align}
Then the observation function of the whole system could be denoted as,
\begin{align}\label{transition}
	Z\left(O|S',\Phi\right) = \psi\left(O|S_U^{t,'}, S_B^{t,'}\right)
	% \prod_{i = 1,i \ne t}^{i = N_A}\mbox{P}\left(S_B^{i,'}, S_U^{i,'}|S_U^i, S_B^i, \Phi\right)
\end{align}
The reward is easy to define, if the access succeeds, then \(R = 1\), else \(R= 0\).
Then the value function of a single state is, 
\begin{equation}
\begin{aligned}
	V_t^\Phi\left(S\right) = R\left(S,\Phi\right) +\gamma\sum\limits_{S'}T\left(S'|S,\Phi\right)V_{t-1}^\pi\left(S'\right),
\end{aligned}%\sum\limits_{O'}Z\left(O|S',\Phi\right)
\end{equation}
where \(\pi\) denotes the optimal action in that state.
As no full knowledge is held for the SU, we use a belief vector to denote the system belief \(\beta\),
\begin{align}
	\beta = \lbrack \beta\left(S = 1\right),\,\beta\left(S = 2\right),\,...,\,\beta\left(S = N_S\right)\rbrack	
\end{align}
Then the particular policy tree with a certain belief \(\beta\) is given as
\begin{equation}
\begin{aligned}
	V_t^\Phi\left(\beta\right) = & \sum\limits_{S}R\left(S,\Phi\right)\beta\left(S\right) +\\
	&	\gamma\sum\limits_{S}\sum\limits_{S'}\beta\left(S\right)T\left(S'|S,\Phi\right)V_{t-1}^\pi\left(S\right),
\end{aligned}
\end{equation}
For simplicity, if we already know all the value function of at the time \(t-1\) during iterations,
a value vector \(\alpha_t^\Phi\) that correspond with each state could be computed and used.
\begin{align}
	\alpha_t^\Phi = \lbrack \alpha_t^\Phi\left(S = 1\right),\,
	\alpha_t^\Phi\left(S = 2\right),\,...,\,\alpha_t^\Phi\left(S = N_S\right)\rbrack
\end{align}
Note that for the same \(\Phi\) and \(t\), there are mutiple \(\alpha\) vector.
The value function could be re-written as
\begin{align}
	V_t^\Phi\left(\beta\right) = \sum\limits_{S}\beta\left(S\right)\alpha_t^\Phi,
\end{align}
and the maximum value function and optimal action is 
\begin{equation}
\begin{aligned}
	V_t^\pi\left(\beta\right) =
	\underset{\alpha_t^\Phi}{\max}\sum\limits_{S}\beta\left(S\right)\alpha_t^\Phi,\\
	\pi\left(\beta\right) =
	\arg\underset{\alpha_t^\Phi}{\max}\sum\limits_{S}\beta\left(S\right)\alpha_t^\Phi,\\
\end{aligned}
\end{equation}
However, the corresponding optimal policy is not as easy as it seems to be, as the \(\beta\) has a continous value.
But fortunately, note that the \(V_t\) could be regarded as the function value of \(\beta\) in a hyper coordinate, 
the axis of which is the 
\(\beta = \lbrack \beta\left(S = 1\right),\,\beta\left(S = 2\right),\,...,\,\beta\left(S = N_S\right)\rbrack\)
As the function value \(V_t\) is piecewise linear with the \(\beta_t\), 
each set of \(\alpha_t\) vector could be regarded as a set of parameters of a hyper linear function.
So there is a dominanted hyperplane structure in the model, i.e., 
the continous belief space is divided into several hyperplane, each of them dominanted by one policy.
The partition of dominanted hyperplane in time \(t\) could be calculated given all the \(\alpha_{t-1}\),
the details of algorithm for solving the partition could be find in a well written tutorial \cite{}.
At the end of the time slot, the SU will update its belief vector according to its observation.
\begin{align}
	\beta'\left(S'|\Phi, O\right) = \frac{\sum\limits_{O}Z\left(O|S',\Phi\right)\sum\limits_{S}
	T\left(S'|S,\Phi\right)\beta\left(S\right)}
	{\sum\limits_{S'}\sum\limits_{O}Z\left(O|S',\Phi\right)\sum\limits_{S}T\left(S'|S,\Phi\right)\beta\left(S\right)}
\end{align}
\subsection{Suboptimal Access Policy}
The optimal POMDP solution could be calculated off-line and 
computed within seconds when the number of states is small. 
However, the use of POMDP method is limited in two situations: when \(N_A\) is massive, 
and when the environment parameters, like sun light coefficients \(\mu_S\), \(\sigma\), 
the birth rate \(\lambda\) and death rate \(\mu\) of PUs, changes drasticly.\\
The optimal strategy in the original POMDP formulation tries to maximize the reward, 
which is the number of success access during a fix-length time slots \(N_T\).
We propose a dual perspective of solving the problem, where we focus on the overall harvested energy.
We name it Energy Based (EB) Method.
The original problem is reformulated as,
\begin{equation}
\begin{aligned}
	\underset{\Phi_t,t=0,\,...,\,N_T}{\max}\sum\limits_{t=1}^{N_T}\mbox{E}\lbrack\mbox{H}\left(S^T, \Phi_t\right)\rbrack,\\
\end{aligned}
\end{equation}
where the \(\mbox{H}\left(S^T, \Phi_T\right) = \sum\mbox{min}\left(E_H,\,E_T+B_M-Q_B\right)\)
is the overall harvesting energy of all the BSs combined.
Now a suboptimal could be proposed as follow,
trying to maximize the system's ability to harvest energy in the next time slot.
\begin{equation}
\begin{aligned}
	\Phi_t\left(S\right) = \arg{\max}\,\,\mbox{E}\lbrack\mbox{H}\left(S^{t,'}, \Phi_t\right)\rbrack,\\
\end{aligned}
\end{equation}
There is good rationality in using an EB perspective.
First of all, when the sun density \(\mu_S\) is strong, 
we could assume few PU will be forced to leave the BS before its original death time.
Second, in most cases, the transmitting power \(E_T = \Upsilon_T(S_U, Q_B, \Phi)\) 
remains quasi linear with the number of user requesting service.
Third, the EB method is a unselfish method, which would sacrifice some reward, 
but will protect the PUs and increase overall utility.
Fourth, the EB method could adjust to quick changes of the environmental parameters, 
which will be show in the journal version of this work.
\section{Simulation Results}
\bibliographystyle{IEEEtran}
\bibliography{Ref}
\end{document}




In recent years, Super WiFi using IEEE 802.11af standard, was announced by the Federal Communications Commission (FCC), aiming to expand the coverage of wireless network access. Since the access points (AP) of Super WiFi will be deployed everywhere, the associated infrastructure including backhaul and energy supply system have to be simplified as much as possible to decrease the deployment cost. Wireless backhual has been proved to be able to substitute traditional wired backhaul \cite{30}. While for the power supply problem, constant battery replacement comes with high labor cost and additional administrative fee. The recent energy harvesting (EH) technology provide with an ideal solution for the power supply problem in Super WiFi. Wireless network service supplied by EH is equipped with devices that could harvest ambient energy including piezoelectric, thermal, solar energy etc. Some preliminary studies have been done on the applications of EH wireless network service, addressing the modification that is required to the current WLAN standard for better support \cite{27}. \\
%%%%%%
\indent In Super WiFi networks, when confronted with many APs, how a user makes the AP selection is a critical issue. Especially, when it comes to AP supplied by solar energy, the energy condition of each AP should be taken into account, besides quality of service (QoS). In order to solve the selection problem, many researches have been done in the literature. Markov Decision Process (MDP) has been used in AP selection game, e.g., some strategies, challenges and solutions were summarized in \cite{23}. As AP selection is a typical game problem \cite{22}, game theory has also been used as an effective approach. In \cite{7}, the selection problem was formulated from a pricing game perspective. Meanwhile, the characteristic of negative externality was considered in \cite{5} and \cite{21}, where a maximum number of access users for each AP was set and the utility is closely related with the number of access users. When it comes to AP selection algorithm, a no regret algorithm was designed in \cite{11} to help users to select among distributed APs. For the EH wireless network service, it has been widely studied as in \cite{17,18,19}. In \cite{17}, the authors characterized the indoor light energy availability and studied the energy allocation. In \cite{18}, the routing algorithms were explored from a network topology aspect. An opportunistic routing protocol was studied in \cite{19}, and the authors compared the proposed protocols with non-opportunistic protocols in wireless sensor networks using ambient energy harvesters (WSN-HEAP).\\
\indent However, all the existing works on network service selection have not considered the situation of APs supplied by ambient power. Apparently, for Super WiFi, EH is the most feasible way that could solve the energy supply problem. In addition, the previous works have separated the APs and users, i.e., either focusing on the transmitting strategies of the APs, or the users' access strategies, by assuming the other part is stable or invariant. However, in practical EH based networks, users are changing in number and distribution, as well as the APs' energy status and channel conditions. Therefore, APs and users are affecting each other, and should be considered simultaneously. Considering these problems, we focus on the problem of AP selection game in a wireless network service powered by ambient solar energy, which has a promising industry future as a key component in Super WiFi. Specifically, we formulate the user model and AP's battery model, considering their interaction effect and the battery physical features through MDP formulation. The optimal AP selection strategy is obtained by a designed value iteration algorithm.\\
\indent The rest of this paper is organized as follow. We describe the system model In Section \Rmnum{2}. The optimal selection based on MDP model is presented in Section \Rmnum{3}, including the battery model, user access model and the AP selection algorithm. In Section \Rmnum{4}, we evaluate the performance of our proposed approach. Finally, Section \Rmnum{5} draws the conclusion.

\section{System Model}
In this section, we describe the overall picture of the EH based Super WiFi system and how we model the problem. As shown in Fig. 1, a Super WiFi system consists of \(N\) APs that are supplied by solar energy harvesting is considered. The APs of the system are connected to the server through wireless backhaul. In each time slot, new arriving users can access one of the available APs and stay connected until he/she leaves the associated AP. In the system, we assume that each AP has a maximum number of accessed users \(U_M\), i.e., if a user intends to access a AP whose accessed user number \(U_i = U_{M}\), the access will be denied. For the AP's battery condition, it is quantized into several discrete levels, where the maximum number of levels is denoted as \(B_M\).\\
\indent Since the number of the users accessing one AP and the battery quantity of that AP are different from those of other APs, the QoS one AP provides is also quite different from others'. If we denote the quantized battery level of the \(i^{th}\) AP as \(B_i\), then the utility of the users defined as the individual throughput, can be written by
\begin{equation}
R_i = W_i \log\left(1 + \frac {P_T(U_i, B_i) / N_0} {(U_i-1)P_I/N_0+1}\right) ,
\end{equation}
where \(B_{i} \in \{ j \mid j = 0,1,...,B_M - 1\}\) and \(U_{i} \in \{ i \mid i = 0,1,...,U_M\}\). In the expression, the $W_i$ is the bandwidth, \(N_0\) is the noise power, which we assume to be the same between every AP and users and remain constant all the time. The signal-to-noise power ratio is \({P_T(U_i, B_i)}/ {N_0}\) and the interference-to-noise power ratio is \({P_I} / {N_0}\). The \(P_T(U_i, B_i)\) is the function of the power used to transmit the pockets to users of the AP. The transmitting power of the APs usually increases with the battery level and the user number. The exact expression is beyond the discussion of this work, which is stipulated by the using WLAN standards \cite{29} and studied in \cite{25}.\\
% % %

\section{AP Selection Strategy based on MDP}
\indent In this section, we analyze users' AP selection strategy in EH based Super WiFi networks. Usually, after a user access one specific AP, he/she would stay connected for a period of time, and thus the long-term utility that can be obtained within this period should be considered. Therefore, we use MDP model to formulate this AP selection problem. Here, the system state is defined as both the user number staying in one AP and the remaining battery quantity. In the following, we will first discuss how we formulate the user states and the battery states, and then propose a value iteration algorithm to derive the optimal strategy.
\subsection{Battery Model}
\indent Quantization of the battery quantity to an appropriate number of levels is necessary. There is a trade off between the accuracy of the algorithm and the amount of the states. In our evaluation, four or five levels of battery conditions would return a result that is accurate enough and not having a big algorithm complexity.\\
\begin{figure}[t]
\centering
\includegraphics[width=8.5cm]{figure1.eps}\\
\caption{A schematic map of Super WiFi system, and a common system model of a solar cell.}
\end{figure}
\indent If the duration \(T\) of the time slot is small enough, a battery level could only switch from a level to its adjacent levels at the end of each time slot in a discrete time system. The transition of a level to its neighbour is simultaneously determined by the energy the AP harvests and the energy it consumes when transmitting pockets. As, given the system state, the transmitting power is already known, the following part of this subsection demonstrates the formulation of relationship from the battery level to the harvesting power.\\
\indent The harvesting power is determined by the illumination and the battery level. In this paper, a classic physical model of the photoelectric battery was used, showed in Fig. 1. The harvesting power first increases with the voltage and then decreases to zero when the maximum voltage is reached, as is showed in Fig. 2. Considering the ideal model of the solar cell, the relation between the current and the voltage are given in \cite{2} as
\begin{equation}
\begin{aligned}
J(V) = J_{SC} - J_o\left(e^{\frac{qV} {k_B T_\kappa}} - 1\right),
\end{aligned}
\end{equation}
where the \(k\) and \(T_\kappa\) is the boltzmann's constant and temperature. In the equation, \(J(V)\) is the current density. When illuminated area and the number of cells connected in one harvesting device is constant, the scale factor \(\rho\) from \(J\) to harvesting power is only linearly affected by illumination level \(\zeta\). In a specific period of the day, \(\zeta\) could be regarded as Gaussian distributed with an average illumination intensity. When time slot \(T\) is small enough, the illumination could be regarded as invariable in each time slot. And if we denote the distribution center of \(\rho\) with the average illumination level \(\bar{\zeta}\) as \(\bar{\rho}_\zeta\), the probability density function of \(\rho\) is given as
\begin{equation}
\begin{aligned}
f_{\rho}(x) = \frac {1} {\sqrt {2\pi}\sigma} e ^{-\frac {(x-\bar{\rho}_\zeta)^2} {2 {\sigma}^2}},
\end{aligned}
\end{equation}
where \(\sigma\) is the standard deviation.
The battery quantity of the \(i^{th}\) battery level \(B_i\) is approximated as \(\overline {B_i}\) and from the electrical formula between voltage and energy quantity in a capacitor, we have \(V = \sqrt{ \frac{2 \overline{B_i}} {C}}\). Thus the function of harvesting power with the voltage is is given as
\begin{equation}
\begin{aligned}
P_H(B_i) = \rho\left(J_{SC} - J_o\left(e^{\frac{qV / C} {k_B T}} - 1\right)\right) \cdot V,
\end{aligned}
\end{equation}
Thus the relation between the battery level and the harvesting power is formulated. \\
\indent For denotation convenience, the \(k^{th}\) APs in the Super Wifi system is refer to as AP \(k\), where \(k \in \{ 1,2,3,...,N \}\). As the transmitting power and harvesting power are known, spontaneously, we could give the transition probability of the battery level of AP \(k\) from level \(i\) to level \(j\) after each time slot, where \(0 \le j \le B_{M} - 1\), and \(0 \le i \le B_{M} - 1\).
\begin{align}
&\beta_{k}(j\mid i,U_k) =\\
&\quad\begin{cases}
\mbox{Pr}\{P_H(k) - P_T (U_k, i) >  {\Delta B_{i,j}}  /T  \}, &\mbox{if $j=i+1$},\\
\mbox{Pr}\{P_T (U_k, i) - P_H(k) > {\Delta B_{j,i}}  /T  \}, &\mbox{if $j=i-1$ },\\
0, &\mbox{if $|j-i|>1$},\\
1 - \sum_{m=0, m\ne i}^{m = B_{M} -1}\beta_{k}(m\mid i), &\mbox{if $j=i$},
\end{cases}\nonumber
\end{align}
where the \(\Delta B_{i,j}\) is the average battery quantity difference between the battery level \(i\) and \(j\).
\begin{figure}
\centering
\includegraphics[width=8.5cm]{figure2.eps}\\
\caption{The current density versus voltage curves and the harvesting power of a cell versus the voltage. The key parameters, \(J_{SC} = 13.364\mbox{mA cm}^{-2}\) and \(V_{OC} = 0.7631\mbox{V}\) is obtained from the work done by Hsiang-Yu Chen \cite{3}.}
\end{figure}
\subsection{User Model}
In the Super WiFi system, two kinds of users are considered. One is self-directed users (SDU) such as the mobile phone users that have several APs in its reach. They could decide which AP to access, and they arrive at a rate of \(\lambda_0\). And the other kind is passive users (PU), for example, the fixed temperature sensors in hospitals. They could only access one specific AP unless the AP is full. For the AP \(k\), its PUs arrive with Poisson arrival distributed rate \(\lambda_{k}\). And the departure rate of all the users in APs is exponential distributed with parameter \(\mu\). To make the formulation clear and concise, we construct a fiducial transition probability from user state \(i\) to user state \(j\) not considering other SDUs and conflicts as follow,
\begin{equation} \psi_{k}(j \mid i) =
\begin{cases}
\lambda_k(1 - i\mu), &\mbox{if $j=i+1,$}\\
i\mu(1 - \lambda_k), &\mbox{if $j=i-1,$}\\
0, &\mbox{if $|j-i|>1.$}\\
\end{cases}
\end{equation}
The total system state including all the APs inside is denoted as a vector \(S = (B_1,U_1,B_2,U_2,...B_N,U_N)\). Given the system states, the SDUs will choose their optimal strategy denoted as \(\pi(S)\). When the conflict of PUs happens, a re-access is carried out. When a PD failed to access an AP because it is full, he/she will be immediately send to a available new AP chosen by the Super Wifi system. A greedy re-access protocol is used by the system. The stand-by AP is given by
\begin{equation}
\begin{aligned}
\xi(S)=\arg\underset{i < B_{M}}{\max}\, W_i \log\left(1 + \frac {P_T(U_i + 1, B_i) / N_0} {(U_i)P_I/N_0+1}\right).
\end{aligned}
\end{equation}
The probability of conflict of a full AP \(k\) in the next time slot is \((1-U_M \mu)\lambda_k\). Thus, the fiducial transition probability could renew into the real transition probability as below:
\begin{equation}
\begin{aligned}
\psi_{\xi(S)}(i+1 \mid i)= &\psi_{\xi(S)}(i+1 \mid i) +\\
&\sum_{m = 1,U_m = U_M}^{N}(1-U_M \mu) \lambda_m,\\
\end{aligned}
\end{equation}
\begin{equation}
\begin{aligned}
\psi_{\pi(S)}(i+1 \mid i) = \psi_{\pi(S)}(i+1 \mid i) + \lambda_0,
\end{aligned}
\end{equation}
and the probability that the system would remain the same is calculated as,
\begin{equation}
\begin{aligned}
\psi_{\pi(S)}(i \mid i) = 1 - \sum\nolimits_{m=0, m\ne i}^{m = B_{M} -1}\psi_{k}(m\mid i).\\ %\qquad\mbox{if $j=i$}.\\
\end{aligned}
\end{equation}
The final system transition probability is defined as follow:
\begin{equation}
\begin{aligned}
P(S'\mid S) = \prod\limits_{k=1}^N \psi_{k}(U_k' \mid S)\beta_{k}(B_k' \mid S).
\end{aligned}
\end{equation}
\subsection{Expected Utility}
A MDP over infinite horizon is considered to obtain the utility of the users. As the PUs are heteronomous, we consider the selection of AP for SDUs. A SDU can not change his AP after accessing, and we use \(\gamma\) to denote the accessed AP. The utility until he leaves is denoted as \(V_{\gamma}(S_0)\). Then his expected utility could be given as,
\begin{equation}
\begin{aligned}
V_{\gamma}(S_0) = E \left(\sum\limits_{t=0}^\infty (1-\tau)^t R_{\gamma}(S_t)\Bigg| S_0\right).
\end{aligned}
\end{equation}
Here \(1 - \tau\) is the discount factor. For iteration algorithm, the equation above could also be written in a recursive way \cite{26},
\begin{equation}
\begin{aligned}
V_{\gamma}(S_0) = R_{\gamma}(S_0) + (1-\tau) \sum\limits_{S'}P_{\gamma}(S'|S_0)V_{\gamma}(S').
\end{aligned}
\end{equation}
Given that the SDU accesses and stays in \(\gamma\), the fiducial transition probability \(\psi_{k}(j \mid i)\) is not the same as the equation (5), but
\begin{equation}
\psi_{k}(j \mid i, \gamma) =
\begin{cases}
\lambda_k(1- (i - 1)\mu), &\mbox{if $j=i+1,k = \gamma$},\\
\lambda_k(1- i\mu), &\mbox{if $j=i+1,k \neq \gamma$},\\
(1 - \lambda_k)(i - 1)\mu, &\mbox{if $j=i-1, k = \gamma $},\\
(1 - \lambda_k)i\mu, &\mbox{if $j=i-1, k \neq \gamma$},\\
0, &\mbox{if $|j-i|>1$}.
\end{cases}
\end{equation}
To solve the real transition probability, the strategy of newcome SDUs is needed. In system \(s\), the strategy of newcome SDUs is given as,
\begin{equation}
\begin{aligned}
\pi_s = \arg \underset {k \leq N}{\max} \ V_{k}(s'),
\end{aligned}
\end{equation}
where \(s'\) is the system state that is different to \(s\) only in that \(U_k' = U_k + 1\), as the access of the AP \(k\) will lead the system state from \(s\) to \(s'\). With the optimal strategy obtained, the transition probability could be calculated by applying equations (8) to (10). In order to solve the MDP iteration, we use the algorithm showed in the table above, which is a revised form of value iteration, first used in \cite{5}.\\
\indent As no SDUs could gain a higher expected utility by not choosing the optimal strategy in the profile, the optimal strategy meets a Nash equilibrium.
\begin{algorithm}[t]
\caption{Value iteration algorithm}
\label{alg}
\begin{algorithmic}
\STATE /********Initialization*********/
\STATE Initialize the \(V_k^{(0)}(s)\leftarrow 0\) for all \(s \in \mathcal{S}\), \(k \in \mathcal{AP}\)
\STATE Initialize the \(\pi^{(0)}(s)\leftarrow 1\) for all \(s \in \mathcal{S}\), \(k \in \mathcal{AP}\)
\STATE /***********Iteration**********/
\WHILE{\(\max\limits_s \mid V_k^{(t)}(s) - V_k^{(t+1)}(s)\mid >\epsilon\)}
\STATE for all \(s \in \mathcal{S}\), \(k \in \mathcal{AP}\)
\STATE {\( V_k^{t+1}(s) \leftarrow R_{k}(s)\ \)+ \\
\(\qquad\qquad\qquad(1-\tau)\sum\limits_{s'}P_{k}(s'|s,\pi^t)V_k^t(s')\).}
\STATE \(\pi^{(t+1)}(s) \leftarrow \arg\underset{\gamma \leq N}{\max}\, V_{\gamma}(\tilde{s})\),\\
            where \(s'\) is the system state different \\
            \qquad\qquad\qquad from \(s\) only in that \(U_k' = U_k + 1\)
\ENDWHILE
\STATE /***********Output***********/
\STATE \( \pi^{\ast}(s) \leftarrow \pi^{t+1}(s)\).
\STATE \( V^{\ast}(s) \leftarrow \underset{\gamma \leq N}{\max}\, V_{\gamma}(\tilde{s})\).
\end{algorithmic}
\end{algorithm}
\section{Simulation Results}
\begin{figure}
\centering
\includegraphics[width=8.5cm]{newResult2.eps}\\
\caption{The convergence of the MDP value iteration.}
\end{figure}
In this section, the convergence and effectiveness of the proposed algorithm are illustrated. In the effectiveness simulation, myopic strategy and random strategy are used as comparations. The myopic strategy is the strategy that SDU users will access the AP that could provide the maximum immediate reward, namely, \(\pi_s^{myopic} = \arg \underset {k \leq N}{\max} \ R_{k}(s)\). The random strategy is that the SDUs will access random AP that is in the Super WiFi system, i.e. \(\mbox{Pr}\{\pi_s^{rand} = i\} = \frac {1} {N}\). According to model and analysis in the Section \Rmnum{2}, \Rmnum{3}, we ascertain several important coefficients. Firstly, same battery quantity differences between the adjacent battery levels is used, i.e. dividing the battery quantity into isometric discrete battery levels. The parameters of the solar cell in \cite{3} is used. As the maximum voltage and storage vary with the number of cells, in this simulation, four power coefficients are of the same order. The charging power \(\max \, P_{H}\), the energy gap between adjacent battery levels divided by the time \(\Delta B / (T \cdot N_0)\), and the power \(P_T\), \(P_I\) are of the same order. We have \(P_T / N_0 = 10\), \(P_I / N_0 = 10\), \(\Delta B / (T \times N_0) = 6\) and \(\max P_{H} / N_0 = 6,...,10\) when \(\bar{\zeta} = 6 \zeta_{Unit},...,10\zeta_{Unit}\). The simulation was operated 10000 times to avoid random error. \\
\indent In Fig. 3, the convergence of the MDP value iteration was illustrated. The Fig. 3(a) is a picture illustrating the sum of \(\mid V_k^{t+1}(s) - V_k^{t}(s)\mid^2\), for all \(s \in \mathcal{S}\), \(k \in \mathcal{AP}\). And Fig. 5(b) illustrates the number of different AP selections between adjacent iterations. From the \(15^{th}\) iteration, the selection strategy remains the same after every iteration and the result is not drawn in the Fig. 3(b). The figure shows that our MDP converges in an exponential way.\\
\indent In Fig. 4, the performance versus arrival rate of the SDUs \(\lambda_0\) and the leaving rate \(\mu\) is evaluated. The coefficients are \(\lambda_1 = \lambda_2 = \lambda_3 = 0.1\),\ \(N = 3\) APs, with \(U_i = 0, 1,...,4\) and \(B_M=4\), \(\zeta = 10 \zeta_{Unit}\). In Fig. 4(a) \(\mu = 0.1\), and in Fig. 4(b) \(\lambda_0 = 0.1\). It is showed in the figure that the proposed strategy has an evident advantage over the myopic one, and random strategy performs the worst. In Fig. 4(a), the myopic utility is set to 1 and normalized the other two accordingly. In Fig. 4(b), the utility decreases with departure rate, as users stay shorter in the system. When \(\mu\) gets bigger, the proposed strategy has less advantage over the myopic. Because a shorter stay time results in more importance of immediate reward.\\
\indent Fig. 5 shows the performance of different strategies under different illumination level. The parameters are \(\lambda_0 = \lambda_1 = \lambda_2 = \lambda_3 = 0.1\), \(\mu = 0.1\), \(N = 3\) APs, and \(U_i = 0, 1,...,4\), \(B_M=4\). All the strategy has a trend of expecting a higher utility as there is more solar energy. And the proposed strategy has a stable advantage over the other two.\\
\begin{figure}
\centering
\includegraphics[width=8.5cm]{newResult1.eps}\\
\caption{The performance of the proposed strategy, myopic strategy and the random strategy is evaluated versus the arrival rate of the SDUs and the departure rate of the users in the APs.}
\end{figure}
\section{Conclusion}
In this paper, we have studied the selection problem of AP in Super WiFi powered by solar energy with negative externality. We in this work formulate the user number and the battery level of each AP as Markov states. We successfully construct the transition probability and use value iteration algorithm to solve the optimal selection strategy in MDP. The result of our proposed algorithm shows that our formulation and algorithm are correct and reliable, with the proposed strategy having a stable and evident outperformance. Our work could be instructive for the development of Super WiFi, providing an effective selection strategy.
\begin{figure}
\centering
\includegraphics[width=8.5cm]{newResult3.eps}\\
\caption{The different performance of three strategies under different illumination levels.}
\end{figure}

